\documentclass[10pt,a4paper,openany]{book}
\usepackage[normalem]{ulem}
\usepackage[boldfont,slantfont,CJKmath=true]{xeCJK}
\usepackage{tocloft}
\usepackage{amsmath,amssymb,amsthm}
\usepackage{unicode-math}
\usepackage[left=0.6in,right=0.6in,top=0.8in,bottom=0.8in]{geometry}
\usepackage{fancyhdr}
\usepackage{fontspec}
\usepackage{enumitem}
\usepackage{parskip}
\usepackage{graphicx}
\usepackage{float}
\usepackage{tabularx}
\usepackage{booktabs}
\usepackage{tabu}
\usepackage{longtable}
\usepackage[colorlinks,linkcolor=blue,urlcolor=blue,anchorcolor=blue,citecolor=blue,breaklinks=true,hyperfootnotes=true,unicode]{hyperref}
\usepackage[cache=false]{minted}
\usepackage{fvextra}
\usepackage{framed}
\usepackage{footnotebackref}
\usepackage{xpatch}
\usepackage{titlesec}
\usepackage[framemethod=TikZ]{mdframed}
\usepackage{datetime}
\usepackage{caption}

% additional CJK characters for '←' and '→'
\xeCJKsetcharclass{"2190}{"2192}{1}% 1: CJK
\xeCJKsetup{xCJKecglue}

% date in Chinese
\renewcommand{\today}{\number\year 年 \number\month 月 \number\day 日}

\title{OI Wiki (Beta)}
\author{OI Wiki 项目组}
\date{\today}
\titleformat{\chapter}[display]{\Huge\bfseries}{第~\thechapter~章}{1em}{}

\setmainfont{STIX Two Text}
\setsansfont{CMU Sans Serif}
\setmonofont{CMU Typewriter Text}
\setmathfont[math-style=ISO,BoldFont={XITS Math Bold},BoldItalicFont={XITS Math Bold}]{STIX Two Math}

\setCJKmainfont[BoldFont={Source Han Serif SC Bold},ItalicFont={KaiTi},BoldItalicFont={Source Han Serif SC Heavy}]{Source Han Serif SC}
\setCJKsansfont{Source Han Sans SC}
\setCJKmonofont{FangSong}

\pagestyle{fancy}

\renewcommand{\chaptername}{第~\thechapter~章}
\renewcommand{\chaptermark}[1]{\markboth{\chaptername\quad #1}{}}
\renewcommand{\sectionmark}[1]{\markright{\thesection\quad #1}}

\fancyhf{}
\fancyhead[LE,RO]{\thepage}
\fancyhead[RE]{\it\nouppercase\leftmark\rm}
\fancyhead[LO]{\it\nouppercase\rightmark\rm}

\setlength{\headheight}{12.5pt}

\setenumerate[1]{itemsep=0pt,partopsep=0pt,parsep=\parskip,topsep=5pt}
\setitemize[1]{itemsep=0pt,partopsep=0pt,parsep=\parskip,topsep=5pt}
\setdescription{itemsep=0pt,partopsep=0pt,parsep=\parskip,topsep=5pt}
\renewcommand{\figurename}{图}
\renewcommand{\tablename}{表}
\renewcommand{\contentsname}{目录}
\allowdisplaybreaks[4]
\definecolor{anti-flashwhite}{rgb}{0.95, 0.95, 0.96}

% mdframed
\definecolor{warning-orange}{rgb}{0.9922,0.9567,0.9098}
\definecolor{info-blue}{rgb}{0.9373,0.9529,1.0}
\definecolor{Red}{rgb}{1,0,0}
\definecolor{Blue}{rgb}{0,0,1}
\newenvironment{details}[2]{\begin{mdframed}[backgroundcolor=#1,hidealllines=true]#2\end{mdframed}\relax}{\vspace{1em}}

\captionsetup[figure]{labelsep=space}
% \captionsetup[table]{labelsep=space}

% framed settings
\definecolor{shadecolor}{rgb}{0.95, 0.95, 0.96}

\newcommand{\hyref}[2]{\href{#1}{\hyphenchar\font="200B #2}}

% auto-wrap typewriter style
\newcommand{\hytt}[1]{\texttt{\hyphenchar\font="200B #1}}

% minted settings
\setminted{bgcolor=anti-flashwhite}

% customize TOC
\cftsetpnumwidth{2em}
\cftsetindents{chapter}{0em}{2em}
\cftsetindents{section}{2em}{3em}
\cftsetindents{subsection}{5em}{4em}

% image related
\graphicspath{{images/}}
\makeatletter
\def\maxwidth#1{\ifdim\Gin@nat@width>#1 #1\else\Gin@nat@width\fi}
\makeatother

% URL auto line break
\makeatletter
\def\UrlAlphabet{%
	\do\a\do\b\do\c\do\d\do\e\do\f\do\g\do\h\do\i\do\j%
	\do\k\do\l\do\m\do\n\do\o\do\p\do\q\do\r\do\s\do\t%
	\do\u\do\v\do\w\do\x\do\y\do\z\do\A\do\B\do\C\do\D%
	\do\E\do\F\do\G\do\H\do\I\do\J\do\K\do\L\do\M\do\N%
	\do\O\do\P\do\Q\do\R\do\S\do\T\do\U\do\V\do\W\do\X%
	\do\Y\do\Z%
}
\def\UrlDigits{\do\1\do\2\do\3\do\4\do\5\do\6\do\7\do\8\do\9\do\0}
\g@addto@macro{\UrlBreaks}{\UrlOrds}
\g@addto@macro{\UrlBreaks}{\UrlAlphabet}
\g@addto@macro{\UrlBreaks}{\UrlDigits}
\makeatother

% blockquote 2em indent
\patchcmd{\quotation}{1.5em}{2em}{}{}

\begin{document}
	\maketitle
	\renewcommand{\baselinestretch}{1.2}\normalsize % 行距设置
	\setlength{\parskip}{0\baselineskip}
	\renewcommand{\headrulewidth}{0mm} % 页眉横线宽度
	\makeatletter
	\let\@afterindentfalse\@afterindenttrue
	\@afterindenttrue
	\makeatother
	\setlength{\parindent}{2em} % 中文缩进两个汉字位
	\setlength{\tabcolsep}{2pt} % 表格列间距 2pt
	\frontmatter
	\tableofcontents
	\clearpage
	\mainmatter
	% Generated by remark-latex
\label{sect:example}
\chapter*{Heading 1}
\chapter*{Heading 2}
\section*{Heading 3}
\subsection*{Heading 4}
\subsubsection*{Heading 5}
\paragraph*{Heading 6}
\chapter*{Emphasis, strong emphasis, and very strong emphasis}
\par An \emph{emphasis}. A {\bfseries strong emphasis}. A {\bfseries \emph{very strong emphasis}}.
\par More \emph{emphasis} test. Even more {\bfseries strong emphasis} test. Even more {\bfseries \emph{very strong emphasis}} test.
\par Or try {\bfseries \emph{mixed style very strong emphasis}}. This is an \emph{{\bfseries alternative}}. This is another \emph{{\bfseries alternative}}. This is one more {\bfseries \emph{alternative}}.
\chapter*{Crossed-out texts}
\par This text is \sout{c}​\sout{r}​\sout{o}​\sout{s}​\sout{s}​\sout{e}​\sout{d}​\sout{ }​\sout{o}​\sout{u}​\sout{t}.
\chapter*{Escape characters, ordered and unordered lists}
\par These are all 10 escape characters reserved by LaTeX:
\begin{itemize}
\item  \_
\item  \&
\item  \textbackslash{}
\item  \textasciitilde{}
\item  \textasciicircum{}
\item  \#
\item  \%
\item  \{
\item  \}
\item  \$
\end{itemize}
\par This is an ordered lists containing the same items:
\begin{enumerate}
\item  \_
\item  \&
\item  \textbackslash{}
\item  \textasciitilde{}
\item  \textasciicircum{}
\item  \#
\item  \%
\item  \{
\item  \}
\item  \$
\end{enumerate}
\par With starting numbering other than 1:
\begin{enumerate}
\setcounter{enumi}{5}
\item  \_
\item  \&
\item  \textbackslash{}
\item  \textasciitilde{}
\end{enumerate}
\par This is another test:
A paragraph containing HTML escape entity and LaTeX\_escape\_characters: AT\&T; © < 3 \& 5 = 10 \textbackslash{} \textasciitilde{}
\chapter*{Inline code and code blocks}
\par This is an \hytt{​i​n​​l​​i​​n​​e​​ ​​c​​o​​d​​e​}, printed with typewriter typeset.
\par This is a \hytt{​c​o​​d​​e​​ ​​b​​l​​o​​c​​k​} written in C++:

\begin{minted}[]{cpp}
#include <iostream>
using namespace std;

int main() {
    int var_with_underscore = 0;
	++var_with_underscore;
	cout << var_with_underscore << endl;
	return 0;
}
\end{minted}

\par This is a \hytt{​p​l​​a​​i​​n​​ ​​t​​e​​x​​t​} printed in typewriter style:

\begin{minted}[]{text}
1 2 3
4 5 6
#include <iostream>
int note_that_the_int_before_should_not_highlight;
\end{minted}

\par Or simply omit the \hytt{​t​e​​x​​t​} format indicator:

\begin{minted}[]{text}
This is another plain text block
1 2 3
4 5 6
#include <iostream>
int note_that_the_int_before_should_not_highlight;
this line is soooooooooooooooooooooooooooooooooooooooooooooooooooooooooooooooooo
oooooooooo long that it just cannnnnnnnnnnnnnnnnnnnnnnnnnnnnnnnnnnn't be fitted 
in a single line, even a single word cannot either
\end{minted}

\chapter*{Separators}
\par Below is a horizontal line, half width of the whole page:
\begin{center}\makebox[0.5\textwidth]{\hrulefill}\end{center}
\par And you can write something after this.
\chapter*{Blockquotes}
\par A nested blockquote is as follows:
\begin{shaded}\begin{quotation}
\par This is a blockquote.\begin{shaded}\begin{quotation}
\par An nested blockquote.\par Continue the nested blockquote.\end{quotation}\end{shaded}\par The nested blockquote exits.\end{quotation}\end{shaded}
\chapter*{Line Breaks and footnotes}
\par This is a line,

after a line break,\textsuperscript{\label{endnoteref:examplenote-1}\hyperref[endnote:examplenote]{[1-1]}}

to the third line.\textsuperscript{\label{endnoteref:examplenote-2}\hyperref[endnote:examplenote]{[1-2]}}
\par Note that the paragraph above contains two note marks pointing to the same endnote.
\par This is a new paragraph.
\chapter*{Links}
\par \hyref{https://example.com}{​a​l​​p​​h​​a​}
\par \hyref{https://example.com}{​h​t​​t​​p​​s​​:​​/​​/​​e​​x​​a​​m​​p​​l​​e​​.​​c​​o​​m​}
\par \hyref{https://example.com/a\_very\_very\_very\_looooooooooooooooooooooooooooooooooooooooooooooooooooooooooooooooooooooooooooooooong\_link}{​h​t​​t​​p​​s​​:​​/​​/​​e​​x​​a​​m​​p​​l​​e​​.​​c​​o​​m​​/​​a​​\_​​v​​e​​r​​y​​\_​​v​​e​​r​​y​​\_​​v​​e​​r​​y​​\_​​l​​o​​o​​o​​o​​o​​o​​o​​o​​o​​o​​o​​o​​o​​o​​o​​o​​o​​o​​o​​o​​o​​o​​o​​o​​o​​o​​o​​o​​o​​o​​o​​o​​o​​o​​o​​o​​o​​o​​o​​o​​o​​o​​o​​o​​o​​o​​o​​o​​o​​o​​o​​o​​o​​o​​o​​o​​o​​o​​o​​o​​o​​o​​o​​o​​o​​o​​o​​o​​o​​o​​o​​o​​o​​o​​o​​o​​o​​o​​o​​o​​o​​n​​g​​\_​​l​​i​​n​​k​}
\par Here we define and use a link reference: \hyref{https://example.com}{​a​l​​p​​h​​a​}

\chapter*{Images}
\par gif:
\par \begin{figure}[H]
\centering
\includegraphics[width=\maxwidth{0.7\linewidth}]{{exampleg}}
\caption{}\end{figure}
\par svg:
\par \begin{figure}[H]
\centering
\includegraphics[width=\maxwidth{0.7\linewidth}]{{examples}}
\caption{}\end{figure}
\par png:
\par \begin{figure}[H]
\centering
\includegraphics[width=\maxwidth{0.7\linewidth}]{{examplep}}
\caption{png}\end{figure}
\par jpg:
\par \begin{figure}[H]
\centering
\includegraphics[width=\maxwidth{0.7\linewidth}]{{examplej}}
\caption{jpg}\end{figure}
\chapter*{Tables}
\par Normal table:
\begin{longtabu}to\linewidth[c]{X[2,c,m]X[2,l,m]X[2,r,m]X[2,c,m]}
\toprule
h1 & h2 & h3 & h4 \\
\midrule
\endfirsthead
\toprule
h1 & h2 & h3 & h4 \\
\midrule
\endhead
\bottomrule
\endfoot
\bottomrule
\endlastfoot
1 & 2 & 3 & 4 \\
\specialrule{0em}{0.4em}{0.4em}5 & 6 & 7 & 8 \\
\end{longtabu}
\par Table with different items contained in rows:
\begin{longtabu}to\linewidth[c]{X[2,c,m]X[2,l,m]X[2,r,m]X[2,c,m]}
\toprule
h1 & h2 & h3 & h4 \\
\midrule
\endfirsthead
\toprule
h1 & h2 & h3 & h4 \\
\midrule
\endhead
\bottomrule
\endfoot
\bottomrule
\endlastfoot
1 & 2 & 3 & 4 \\
\specialrule{0em}{0.4em}{0.4em}5 & 6 & 7 \\
\end{longtabu}
\par Table with no alignment specified and one line containing too many cells:
\begin{longtabu}to\linewidth[c]{X[2,c,m]X[2,c,m]X[2,c,m]X[2,c,m]}
\toprule
h1 & h2 & h3 & h4 \\
\midrule
\endfirsthead
\toprule
h1 & h2 & h3 & h4 \\
\midrule
\endhead
\bottomrule
\endfoot
\bottomrule
\endlastfoot
1 & 2 & 3 & 4 \\
\specialrule{0em}{0.4em}{0.4em}5 & 6 & 7 & 8 \\
\end{longtabu}
\chapter*{Math}
\par This is a math test: $1 + 1 = 2$.
\par This is a display math:
\begin{equation*}
\sum_{n=1}^{+\infty}\dfrac{1}{n^2}=\dfrac{\pi^2}{6}
\end{equation*}

\chapter*{Endnotes}

\begin{enumerate}
\renewcommand{\labelenumi}{[\theenumi]}
\item\label{endnote:examplenote}a footnote. \hyperref[endnoteref:examplenote-1]{[1-1]} \hyperref[endnoteref:examplenote-2]{[1-2]}
\end{enumerate}

\end{document}
